\documentclass[10pt,landscape]{scrartcl}

\usepackage[landscape]{geometry}
\usepackage{hyperref}
\hypersetup{colorlinks=true}
\usepackage{multicol}
\usepackage{calc}
\usepackage{ifthen}
\usepackage{amsmath}
\usepackage{tikz}
\usetikzlibrary{quantikz2}

\title{Quantum Computing Midterm 1 Cheatsheet}
\author{Aly Cerruti}
\date{2025-10-09}

% This sets page margins to .5 inch if using letter paper, and to 1cm
% if using A4 paper. (This probably isn't strictly necessary.)
% If using another size paper, use default 1cm margins.
\ifthenelse{\lengthtest { \paperwidth = 11in}}
{ \geometry{top=.5in,left=.5in,right=.5in,bottom=.5in} }
{\ifthenelse{ \lengthtest{ \paperwidth = 297mm}}
{\geometry{top=1cm,left=1cm,right=1cm,bottom=1cm} }
{\geometry{top=1cm,left=1cm,right=1cm,bottom=1cm} }
}

% Turn off header and footer
\pagestyle{empty}

% Don't print section numbers
\setcounter{secnumdepth}{0}

\setlength{\parindent}{0pt}
\setlength{\parskip}{0pt plus 0.5ex}

\begin{document}
    \raggedright
    \footnotesize
    \begin{multicols*}{3}
        % multicol parameters
        % These lengths are set only within the two main columns
        %\setlength{\columnseprule}{0.25pt}
        \setlength{\premulticols}{1pt}
        \setlength{\postmulticols}{1pt}
        \setlength{\multicolsep}{1pt}
        \setlength{\columnsep}{2pt}

        \columnbreak[1]\section{Definitions}\label{sec:definitions}
        
        \(\displaystyle \mathbf{X} = \begin{bmatrix}0 & 1 \\ 1 & 0\end{bmatrix}\)

        \(\displaystyle \mathbf{Y} = \begin{bmatrix}0 & -i \\ i & 0\end{bmatrix}\)

        \(\displaystyle \mathbf{Z} = \begin{bmatrix}1 & 0 \\ 0 & -1\end{bmatrix}\)

        \(\displaystyle \mathbf{H} = \frac{1}{\sqrt{2}} \begin{bmatrix}1 & 1 \\ 1 & -1\end{bmatrix}\)

        \columnbreak[1]\section{Gate Equivalences}\label{sec:gate-equivalences}

        \(\displaystyle \mathbf{X}^2 = \mathbf{Y}^2 = \mathbf{Z}^2 = \mathbf{H}^2 = \mathbf{I}\)

        \(\displaystyle \mathbf{X}\mathbf{Y} = -\mathbf{Y}\mathbf{X} = i\mathbf{Z}\)

        \(\displaystyle \mathbf{Y}\mathbf{Z} = -\mathbf{Z}\mathbf{Y} = i\mathbf{X}\)

        \(\displaystyle \mathbf{Z}\mathbf{X} = -\mathbf{X}\mathbf{Z} = i\mathbf{Y}\)

        \(\displaystyle \mathbf{X}\mathbf{H} = \mathbf{H}\mathbf{Z}\)

        \(\displaystyle \mathbf{H}\mathbf{X} = \mathbf{Z}\mathbf{H}\)

        \columnbreak[1]\section{Circuit Equivalences}\label{sec:circuit-equivalences}

        \begin{quantikz}
            \ghost{Z} & \ctrl{1} & \\
            & \gate{Z} &
        \end{quantikz} \equiv \begin{quantikz}
            & \gate{Z} & \\
            \ghost{Z} & \ctrl{-1} &
        \end{quantikz}

    \end{multicols*}
\end{document}
